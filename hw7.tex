%%!TEX TS-program = xelatexmk
\documentclass[oneside,justified,marginals=raggedouter]{tufte-handout}
\usepackage{fontspec,xltxtra,xunicode}
\usepackage{enumerate}
\usepackage{booktabs}
\usepackage[fleqn]{amsmath}
\usepackage{amssymb}
\usepackage{tikz}
\usetikzlibrary{plotmarks}

\defaultfontfeatures{Mapping=tex-text}
\setmainfont[%Mapping=tex-text,
  BoldFont={Haarlemmer MT Std Bold},
  SlantedFont={Haarlemmer MT Std Italic},
  ItalicFont={Haarlemmer MT Std Italic},
  BoldItalicFont={Haarlemmer MT Std Bold Italic},
  SmallCapsFont={Haarlemmer MT Std: +smcp}
]{Haarlemmer MT Std}
\setsansfont[Scale=MatchLowercase,Mapping=tex-text]{TeX Gyre Heros}
\setmonofont[Scale=MatchLowercase,Ligatures={NoRequired,NoCommon,NoContextual}]{TeX Gyre Cursor}

\DeclareMathOperator*{\argmax}{arg\,max}

% XeTeX workaround for tufte-latex (otherwise need to use xetex or nols options)
\renewcommand{\allcapsspacing}[1]{{\addfontfeature{LetterSpace=20.0}#1}}
\renewcommand{\smallcapsspacing}[1]{{\addfontfeature{LetterSpace=5.0}#1}}
\renewcommand{\textsc}[1]{\smallcapsspacing{\textsmallcaps{#1}}}
\renewcommand{\smallcaps}[1]{\smallcapsspacing{\scshape\MakeTextLowercase{#1}}}

\widowpenalty=1000
\clubpenalty=1000
\raggedbottom
\hyphenpenalty=5000
\tolerance=1000

\title{LING 572 Homework 7}
\author{David McHugh and Chris Curtis}
\date{2 Mar 2013}

\clearpage\relax

\begin{document}
\maketitle


\section{Question 1}

Results on the binary classification task:
\vskip\baselineskip

\begin{tabular}{@{}crlllllll@{}}
\toprule
Expt id & Kernel & gamma & coef0 & degree & total\_sv & Training & Test & Test Acc \\
 &  &  &  &  &  & Acc & Acc & from Q2 \\ \midrule
1 & linear & - & - & - & 535 & 0.997222 & 0.950000 & 0.950000 \\
2 & polynomial & 1 & 0 & 2 & 792 & 0.997222 & 0.920000 & 0.920000 \\
3 & polynomial & 0.1 & 0.5 & 2 & 775 & 0.997222 & 0.965000 & 0.965000 \\
4 & RBF & 0.5 & - & - & 1798 & 0.997222 & 0.500000 & 0.500000 \\
5 & sigmoid & 0.5 & -0.2 & - & 1214 & 0.536667 & 0.405000 & 0.405000 \\

\bottomrule
\end{tabular}


\section{Question 2}

The results in the last two columns are identical.

\end{document}
